% --------------------------------------------------------------
% This is all preamble stuff that you don't have to worry about.
% Head down to where it says "Start here"
% --------------------------------------------------------------
 
\documentclass[12pt]{article}
 
\usepackage[margin=1in]{geometry} 
\usepackage{amsmath,amsthm,amssymb}
\usepackage{algorithm}
\usepackage[noend]{algpseudocode}
\usepackage{listings}
\usepackage{amssymb}


\newcommand{\N}{\mathbb{N}}
\newcommand{\Z}{\mathbb{Z}}
 
\newenvironment{theorem}[2][Theorem]{\begin{trivlist}
\item[\hskip \labelsep {\bfseries #1}\hskip \labelsep {\bfseries #2.}]}{\end{trivlist}}
\newenvironment{lemma}[2][Lemma]{\begin{trivlist}
\item[\hskip \labelsep {\bfseries #1}\hskip \labelsep {\bfseries #2.}]}{\end{trivlist}}
\newenvironment{exercise}[2][Exercise]{\begin{trivlist}
\item[\hskip \labelsep {\bfseries #1}\hskip \labelsep {\bfseries #2.}]}{\end{trivlist}}
\newenvironment{problem}[2][Problem]{\begin{trivlist}
\item[\hskip \labelsep {\bfseries #1}\hskip \labelsep {\bfseries #2.}]}{\end{trivlist}}
\newenvironment{question}[2][Question]{\begin{trivlist}
\item[\hskip \labelsep {\bfseries #1}\hskip \labelsep {\bfseries #2.}]}{\end{trivlist}}
\newenvironment{corollary}[2][Corollary]{\begin{trivlist}
\item[\hskip \labelsep {\bfseries #1}\hskip \labelsep {\bfseries #2.}]}{\end{trivlist}}

\newenvironment{solution}{\begin{proof}[Solution]}{\end{proof}}
 
\begin{document}

\title{MPCS 52060: Parallel Programming Assignment 3}
\author{Justin Cohler}

\maketitle

\subsection*{SAQ 1} 
\textbf{(Ch.1 Exercise 1) For each of the following, state whether it is a safety or liveness property. Identify the bad or good thing of interest.}

\begin{enumerate}
    \item Patrons are served in the order they arrive.

    (Safety) - Sequential ordering is a safety concern.
    \item What goes up must come down.
    
    (Liveness) - Eventual correctness is a liveness concern.
    \item If two or more processes are waiting to enter their critical sections, at least one succeeds.
    
    (Liveness) - This is essentially what deadlock freedom denotes.
   \item If an interrupt occurs, then a message is printed within one second. 
   
   (Safety) - A sequence of events occurs correctly (a message printed in time).
   \item If an interrupt occurs, then a message is printed.
    
    (Liveness) - Eventual correctness of a message being printed is a liveness concern.
    \item The cost of living never decreases.
    
    (Safety) - Sequential correctness is a safety concern.
    \item Two things are certain: death and taxes.
    
    (Liveness) - Eventually, both of these things are realized, therefore this is a liveness concern.
    \item You can always tell a Harvard man.
    
    (Safety) - Correctness in this case regards safety.
\end{enumerate}

\vspace{5mm}

\subsection*{SAQ 2}
\textbf{(Ch.1, Exercise 7) Running your application on two processors yields a speedup of S2. Use Amdahl’s Law to derive a formula for Sn, the speedup on n processors, in terms of n and S2.
}
\vspace{5mm}

Using Ahmdal's Law:
    $$s_2 = 1/(1-p+p/n) = 1/(1-p+p/2) = 1/(1-p/2)$$
    $$s_2 = 1/(1-p/2)$$
    $$s_2 - s_2*p/2 = 1$$
    $$s_2*p/2 = s_2 - 1$$
    
    $$p = 2*(s_2-1)/s_2$$


Now we've solved $p$ in terms of $s_2$. Let's plug $s_2$ back in for $s_n$:
    $$s_n = 1/(1-2(s_2-1)/s_2 + 2(s_2-1)/s_2/n)$$
    $$= 1/(1-2(s_2-1)/s_2 + 2(s_2-1)/ns_2)$$
    
    $$s_n = 1/(1-2(s_2-1)/s_2 + 2(s_2-1)/ns_2)$$


\subsection*{SAQ 3}
\textbf{(Ch.1, Exercise 8) You have a choice between buying one uniprocessor that executes five zillion instructions per second, or a ten-processor multiprocessor where each processor executes one zillion instructions per second. Using Amdahl’s Law, explain how you would decide which to buy for a particular application.
}
Using Amdahls Law, we can define our speedup for ten-processors as follows:
    $$Speedup = 1/(1-p+p/n) = 1/(1-p+p/10)$$
    
Setting the speedup equal to our single processor speed (5 ZOPS) will determine the break-even point for multi-processing.

    $$1 Z/s * Speedup = 5 Z/s$$
    $$= 1 * 1/(1-p+p/10) = 5$$
    $$\rightarrow 1 > 5 - 5p + p/2$$
    $$\rightarrow -4 > p*(-4.5)$$
    $$\rightarrow p > 4/4.5$$
    $$\rightarrow p > 8/9$$


If our application can have a parallelization ratio of greater than 8/9 (meaning 8/9ths of the application can be distributed to cores in parallel), then it becomes worthwhile to buy the ten-processor unit.

\subsection*{SAQ 4}
\textbf{(Ch.2 Exercise 14) The L-exclusion problem is a variant of the starvation-free mutual exclusion problem. We make two changes: as many as L threads may be in the critical section at the same time, and fewer than L threads might fail (by halting) in the critical section. An implementation must satisfy the following conditions:
\begin{itemize}
    \item L-Exclusion: At any time, at most L threads are in the critical section.
    \item L-Starvation-Freedom: As long as fewer than L threads are in the critical section, then some thread that wants to enter the critical section will eventually succeed (even if some threads in the critical section have halted).
\end{itemize}
Modify the n-process Bakery mutual exclusion algorithm to turn it into an L-exclusion algorithm. Do not consider atomic operations in your answer. You can provide a pseudo-code solution or written solution.}

\vspace{5mm}
Below, I've written pseudo-Java that replaces the boolean flag array with an int flag array. Otherwise this is very similar to the Bakery Algorithm provided in the slides.

\begin{lstlisting}
class LExclusion implements Lock {
    int[] flag;
    Label[] label;
    int l;
    
    public LExclusion (int n, int l) {
        this.l = l
        flag = new boolean[n-l];
        label = new Label[n-l];
        for (int i = 0; i < n-l; i++) {
            flag[i] = 0; label[i] = 0
        }
    }
    
    public void lock() {
        for (int i = 1; i < n-l; i++) {
            flag[i] = i;
            label[i] = max(label) + 1;
            while ((label[i],i) > (label[k],k))){
                for (int j = 0; j < n; j++) {
                    if (flag[j] > this.l+1)
                        break;
                }
            }
        }
    }
    
    public void unlock() {
        flag[i] = 0;
    }
    
}
\end{lstlisting}


\end{document}
