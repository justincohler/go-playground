% --------------------------------------------------------------
% This is all preamble stuff that you don't have to worry about.
% Head down to where it says "Start here"
% --------------------------------------------------------------
 
\documentclass[12pt]{article}
 
\usepackage[margin=1in]{geometry} 
\usepackage{amsmath,amsthm,amssymb}
 
\newcommand{\N}{\mathbb{N}}
\newcommand{\Z}{\mathbb{Z}}
 
\newenvironment{theorem}[2][Theorem]{\begin{trivlist}
\item[\hskip \labelsep {\bfseries #1}\hskip \labelsep {\bfseries #2.}]}{\end{trivlist}}
\newenvironment{lemma}[2][Lemma]{\begin{trivlist}
\item[\hskip \labelsep {\bfseries #1}\hskip \labelsep {\bfseries #2.}]}{\end{trivlist}}
\newenvironment{exercise}[2][Exercise]{\begin{trivlist}
\item[\hskip \labelsep {\bfseries #1}\hskip \labelsep {\bfseries #2.}]}{\end{trivlist}}
\newenvironment{problem}[2][Problem]{\begin{trivlist}
\item[\hskip \labelsep {\bfseries #1}\hskip \labelsep {\bfseries #2.}]}{\end{trivlist}}
\newenvironment{question}[2][Question]{\begin{trivlist}
\item[\hskip \labelsep {\bfseries #1}\hskip \labelsep {\bfseries #2.}]}{\end{trivlist}}
\newenvironment{corollary}[2][Corollary]{\begin{trivlist}
\item[\hskip \labelsep {\bfseries #1}\hskip \labelsep {\bfseries #2.}]}{\end{trivlist}}

\newenvironment{solution}{\begin{proof}[Solution]}{\end{proof}}
 
\begin{document}

\title{MPCS 52060: Parallel Programming Assignment 3}
\author{Justin Cohler}

\maketitle

\subsection*{SAQ 1} 
\textbf{(Ch.1 Exercise 1) For each of the following, state whether it is a safety or liveness property. Identify the bad or good thing of interest.}

\begin{enumerate}
    \item Patrons are served in the order they arrive.

    (Liveness)
    \item What goes up must come down.
    
    (Safety)
    \item If two or more processes are waiting to enter their critical sections, at least one succeeds.
    
    (Safety)
   \item If an interrupt occurs, then a message is printed within one second. 
   
   (Liveness)
   \item If an interrupt occurs, then a message is printed.
    
    (Safety)
    \item The cost of living never decreases.
    
    
    \item Two things are certain: death and taxes.
    
    (Safety)
    \item You can always tell a Harvard man.
    
    (Safety)
\end{enumerate}

\vspace{5mm}

\subsection*{SAQ 2}
\textbf{(Ch.1, Exercise 7) Running your application on two processors yields a speedup of S2. Use Amdahl’s Law to derive a formula for Sn, the speedup on n processors, in terms of n and S2.
}

\vspace{5mm}

\subsection*{SAQ 3}
\textbf{(Ch.1, Exercise 8) You have a choice between buying one uniprocessor that executes five zillion instructions per second, or a ten-processor multiprocessor where each processor executes one zillion instructions per second. Using Amdahl’s Law, explain how you would decide which to buy for a particular application.
}

\vspace{5mm}

\subsection*{SAQ 4}
\textbf{(Ch.2 Exercise 14) The L-exclusion problem is a variant of the starvation-free mutual exclusion problem. We make two changes: as many as L threads may be in the critical section at the same time, and fewer than L threads might fail (by halting) in the critical section. An implementation must satisfy the following conditions:
\begin{itemize}
    \item L-Exclusion: At any time, at most L threads are in the critical section.
    \item L-Starvation-Freedom: As long as fewer than L threads are in the critical section, then some thread that wants to enter the critical section will eventually succeed (even if some threads in the critical section have halted).
\end{itemize}
Modify the n-process Bakery mutual exclusion algorithm to turn it into an L-exclusion algorithm. Do not consider atomic operations in your answer. You can provide a pseudo-code solution or written solution.}

\vspace{5mm}

\end{document}
